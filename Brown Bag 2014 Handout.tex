\documentclass{article}
\usepackage{enumerate, amsmath ,amsfonts}
\usepackage[margin=1in]{geometry} %change page margins
\usepackage[english]{SASnRdisplay} %codeblocks in SAS and R. English option gives English captions (otherwise they'll be in Danish)
\usepackage{hyperref} %embed hyperlinks in the pdf
\usepackage{times}

\title{\LaTeX \ 101/201 \\
	\Large{Biostatistics Brown Bag Seminar}}
\author{Anagha Tolpadi and Nick Seewald \\ \small{University of Michigan Department of Biostatistics}}
\date{4 December 2014}

\begin{document}
	\maketitle
	
	$a,b\in \mathbb{R}$
	
	This handout and example files are available online at \href{http://bit.ly/brownbaglatex201}{http://bit.ly/brownbaglatex201}.
	
	\section{Introductory \LaTeX}
	To do most things in \LaTeX, you need to install packages. You do that in the preamble.
	
	This is an exciting new paragraph.
	
	To put things in math mode, surround your equations with \$'s. For example, $\sqrt{4}=2$.
	
	To write out equation's, you can use  equation. equation* will remove the numbers after each line.
	
	\begin{equation}
	\sqrt{4} = 2
	\end{equation}
	
	\begin{equation*}
	\sqrt{4} = 2
	\end{equation*}
	
	If you need to solve through (which will probably be most useful for homework assignments), you can use align.
	
	\begin{align*}
	\sqrt{2+2} &= \sqrt{4} \\
	&= 2
	\end{align*}
	
	Both align and equation will automatically use math mode, so you do not need to use the \$'s.
	\\
	To make lists, you can use an itemize or enumerate environment.
	\begin{enumerate}
		\item
		\begin{itemize}
			\item
			\item
			\end{itemize}
			\item
	\end{enumerate}
			
			This is a table.
			\begin{center}
				\begin{tabular}{c|c|}
					\hline
					a & b \\
					\hline
					c & d \\
					\hline
				\end{tabular}
			\end{center}
	
	\section{R Output: Using \texttt{xtable}}
	\begin{itemize}
		\item \texttt{xtable} is an R package for converting R tables to \LaTeX tables.
		\item To install and load, 
		\begin{Rcode*}
> install.packages(``xtable'')
> library(xtable)
		\end{Rcode*}
		\item Use the \texttt{xtable} function on the table you want to export, then copy and paste the output into your .tex file.
	\end{itemize}
	
	
	\section{SAS Output: Using \texttt{ODS LATEX}}
	\begin{itemize}
		\item ODS (\textbf{O}utput \textbf{D}elivery \textbf{S}ystem) lets you create nice-looking output files containing results from SAS. Examples include HTML (the default), PDF, RTF, etc. ODS also controls graphics output, but, for our purposes, we won't worry about this.
		\item To use ODS, surround your SAS code in \texttt{ODS (TYPE) (...);} and \texttt{ODS (TYPE) CLOSE;}. This does \textit{not} work all the time, but it's generally how things look.
		\item \texttt{ODS LATEX} works like any other ODS method, with some unique syntax.
		\begin{itemize}
			\item Open \texttt{ODS LATEX} with
			\begin{quote}
				\texttt{ODS LATEX FILE= ``<\textit{filename}>.tex'' STYLESHEET=``sas.sty''(url=``sas'');} 
			\end{quote}
			\item Close \texttt{ODS LATEX} with 
			\begin{quote}
				\texttt{ODS LATEX CLOSE;}
			\end{quote}
			\item A full example:
			\begin{SAScode*}
ODS LATEX FILE="fishmeans.tex" STYLESHEET="sas.sty"(URL="sas");
PROC MEANS DATA=sashelp.fish;
	VAR weight length1 length2 length3;
RUN;
ODS LATEX CLOSE;
			\end{SAScode*}
		\end{itemize}
		\item Files (.tex, .sty) will be placed in your current working directory.
	\end{itemize}
	
	\subsection{``Installing'' the SAS Stylesheet, \texttt{sas.sty}}
	\begin{itemize}
		\item Packages in \LaTeX are defined by files called \textit{stylsheets}, which have extension .sty
		\item In \texttt{ODS LATEX} syntax, we call the stylesheet ``sas.sty''. 
		\item Unfortunately, we have to manually install this package, which involves creating some folders on your hard drive.
		\item \textbf{NOTE:} The default \texttt{sas.sty} file forces a lot of changes in document appearance that I dislike. I've modified the default output to allow more flexibility, and to make your \LaTeX look more like \LaTeX. You can find that file at \href{http://bit.ly/brownbaglatex201}{http://bit.ly/brownbaglatex201}.
		\item Instructions for Windows:
		\begin{itemize}
			\item  Move \texttt{sas.sty} to \texttt{C:$\backslash$Local TeX Files$\backslash$tex$\backslash$latex$\backslash$misc$\backslash$}. You'll need to make these folders.
			\item Open MikTeX settings. On the ``Roots'' tab, click ``Add...'', then point to the Local TeX Files folder and click Ok. On the ``General'' tab, click ``Update Formats''.
		\end{itemize}
		\item Instructions for Mac:
		\begin{itemize}
			\item On a Mac, local additions go into the Library/texmf folder of your Home folder.
			\item (~/Library/texmf) You never need to update the file database for local additions to this folder. If the folder doesn't exist, you will need to create it yourself. For a one-click version of this you can use this utility: \href{https://www.msu.edu/~amunn/latex/make-local-texmf.zip}{https://www.msu.edu/$\sim$amunn/latex/make-local-texmf.zip}
			\item If you are running OS 10.7 (Lion) or later, your user Library is unfortunately a hidden folder. It is, however, easy to access it through the Finder's Go menu: hold down the Option key while choosing the Go menu, and the local Library folder will appear in the list. Alternatively if you type Command-Shift-G you can enter the folder path directly (~/Library/texmf)
			\item (This was copied verbatim from StackExchange: \href{http://tex.stackexchange.com/a/10256}{http://tex.stackexchange.com/a/10256}) (Disclaimer: I do not use OS X, so no guarantees that this works.)
			\item Essentially, put \texttt{sas.sty} in /Library/texmf. 
		\end{itemize}
		\item Instructions for Linux:
		\begin{itemize}
			\item You're using Linux, so you're obviously smart enough to figure this out on your own. Sorry!
		\end{itemize}
	\end{itemize}
	
	\subsection{Including SAS Output in Your .tex File}
	\begin{itemize}
		\item Include the \texttt{sas} package in your document's preamble, i.e.,
		\begin{verbatim}
\documentclass{article}
\usepackage{sas}
\begin{document}
		\end{verbatim}
		\item Open your output .tex file (saved in the SAS working directory)
		\item Copy and paste the corresponding \texttt{sastable}s!
	\end{itemize}
	
	\section{Making Presentations in \LaTeX with \texttt{beamer}}
	\begin{itemize}
		\item The most commonly-used way to make a presentation in \LaTeX is to use a \texttt{documentclass} called \texttt{beamer}. The first line of your document should be
		\begin{verbatim}
\documentclass{beamer}
		\end{verbatim}
		\item In the preamble, specify theme, title, author, etc. You can choose both a \texttt{theme} and a \texttt{colortheme}. Think of \texttt{theme} as a layout, and \texttt{colortheme} as a color palette. See \href{http://www.hartwork.org/beamer-theme-matrix/}{http://www.hartwork.org/beamer-theme-matrix/} for a grid of what every default \texttt{theme} and \texttt{colortheme} look like together.
		\item Slides are called \texttt{frame}s, and are environments (like itemize, etc. -- open with \texttt{begin} and close with \texttt{end}.) Inside a frame, you can do most anything you can do in normal \LaTeX.
		\item More in-depth tutorials:
		\begin{itemize}
			\item \href{http://www.math-linux.com/latex-26/article/how-to-make-a-presentation-with-latex-introduction-to-beamer}{http://www.math-linux.com/latex-26/article/how-to-make-a-presentation-with-latex-introduction-to-beamer}
			\item \href{https://en.wikibooks.org/wiki/LaTeX/Presentations}{https://en.wikibooks.org/wiki/LaTeX/Presentations}
		\end{itemize}
	\end{itemize}
	
\begin{thebibliography}{99}
\bibitem{ross}
S. Ross, \textit{A First Course in Probability}, Pearson Prentice Hall, Upper Saddle River, 2010.
\bibitem{mceliece}
R. J. McEliece, \textit{Theory of Information and Coding}, Cambridge University Press, Port Chester, 2001.
\bibitem{jukna}
S. Jukna, \textit{Extremal Combinatorics with Applications in Computer Science}, Springer, New York, 2001.
\bibitem{alon and spencer}
N. Alon \& J. H. Spencer, \textit{The Probabilistic Method}, John Wiley \& Sons, Hoboken, 2008.
\bibitem{radhakrishnan}
J. Radhakrishnan, Entropy and Counting, \textit{IIT Kharagpur, Golden Jubilee Volume, on Computational Mathematics, Modelling and Algorithms}, Ed. J.C. Mishra, Narosa Publishers, New Dehli, 2001.
\bibitem{alon and friedland}
N. Alon \& S. Friedland, The maximum number of perfect matchings in graphs with a given degree sequence, \emph{Electron. J. Combin.} \bf{15} (2008)
\end{thebibliography}

	
\end{document}